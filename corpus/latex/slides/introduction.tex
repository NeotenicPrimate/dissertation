
\begin{itemize}

% In this Science as organization (i.e., organizational strucutre)

% What distinguishes the social organization of "natural" and "social" sciences

% What distinguishes what he call "high-consensus rapid-discovery" science from non-"high-consensus rapid-discovery"?

\item "Form of scientific organization [...] mode of organization" \citep[158-160]{collins1994}

\item "What is distinctive about the social organization of the disciplines that we now take as natural science, and do the social disciplines have (or can they acquire) the conditions that make possible that kind of organization?" \citep[156]{collins1994}

\item What sets appart "\textbf{high-consensus rapid-discovery}" from "\textbf{low-consensus non-rapid-discovery}" science? \citep[158]{collins1994}

\end{itemize}



















% \only<1>{
	
% 	\framesubtitle{Origin of the debate}

% 	Structure \string& dynamics of science
	
% 	\begin{itemize}
		
% 		\item \cite{popper2002}: Methods \string& falsification
		
% 		\item "Recurrent debates about whether one or another of the contemporary social sciences is really a science [...] will cease to be a source of concern not when a definition is found, but when the groups that now doubt their own status achieve \textbf{consensus} about their past and present accomplishments" \citep[160-161]{kuhn2012}
		
% 	\end{itemize}
% }

% \only<2>{

% 	\framesubtitle{Questions}

% 	\begin{itemize}

% 	\item How do the characteristics of natural sciences compare to those of social sciences and humanities in terms of cumulativeness, and what structural factors contribute to these differences?

% 	\item Which fields, if any, can be characterized as "high-consensus, rapid-discovery" \citep{collins1994} and how do these fields differ from others in terms of structure?

% 	\item How do the structural features of different scientific fields vary, and what factors contribute to these differences?

% 	\end{itemize}
% 	\framesubtitle{}

% 	Do different scientific fields differ in their structural features?

% 	Previous litterature:

% 	\begin{itemize}

% 	\item Single field \citep{cahlik2006,yeung2017,qin2020}

% 	\item Temporality \citep{callon2005,beam2014}

% 	\item Qualitative \citep{popper2002,kuhn2012,collins1994,silverstein1991}

% 	\end{itemize}
% }

% \only<3> {
	
% 	\framesubtitle{Function \string& Dynamics}

% 	\fontsize{10}{10}\selectfont
% 	\begin{multicols}{2}
% 	\begin{itemize}
% 		\item “Communities” \citep{kuhn2012}
% 		\item “Tribes” \citep{trowler2001}
% 		\item “Invisible colleges” \citep{desollaprice1966, crane1972, paisley1972}
% 		\item “Teams” \citep{wuchty2007}
% 		\item “Social-intellectual movements” \citep{frickel2005} 
% 		\item “Bandwagons” \citep{shannon1956}
% 		\item “Actor-networks” \citep{latour1988} 
% 		\item “Theory groups” \citep{mullins1973}
% 		\item "Programmes” \citep{lakatos1968,merton1972}
% 		\item “Schools" \citep{radnitzky1971,tiryakian1979}
% 		\item “Specialty” \citep{wray2005}
% 		\item “Parallel processing units” \citep{kornfeld1981}
% 		\item “Invisible brain [regions]” \citep{segev2016}
% 		\item "Complex system" \citep{boulding1956}
% 		\item “Referee system” \citep{zuckerman1971}	
% 	\end{itemize}
% 	\end{multicols}
% }