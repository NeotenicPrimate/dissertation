Exponential Random Graph Models are used to model and test different hypotheses about the 
structural features of social networks. This approach enables us to investigate the different 
local and global properties of a network and establish their importance in explaining its 
formation. By testing for various network properties, such as triangles or the average shortest 
path between nodes, we can determine whether these properties played a significant role in the 
network's formation.

According to the Hammersly and Clifford Theorem (1971), any network model can be expressed in the 
exponential family with counts of graph statistics. The probability of observing a particular graph 
$y$ on $n$ nodes out of the set of all possible graphs on $n$ nodes, $Y$, denoted as $P$, can be 
calculated using a set of network statistics $S$ and corresponding parameters $\theta$.

$$
P_{y,\theta}(Y=y|\theta) = \frac{exp\{\theta^T S(y)\}}{\sum\limits_{y' \in Y} exp\{\theta^T S(y')\}}
$$

The denominator is a normalizing constant that guarantees the distribution adds up to one. 
This constant requires summing over space of possible networks on $n$ nodes. However, the number 
of possible configurations (size of $Y$) grows exponentially with the number of nodes, specifically 
to $2^{(n(n-1)/2)}$ for undirected graphs and $2^{(n(n-1))}$ for directed graphs. Thus, an exact 
computation of this sum is not feasible.

Because of this, it is custommary to use Marcov Chain Monte Carlo method to generate samples. It 
estimate the values of the parameters that maximize the likelihood of the observed network. These 
estimates represent the strength and direction of the effects of various network statistics on the 
likelihood of observing the network.

The above formula can be rewritten in terms of the covariate vector $\theta$:

$$
logit(Y_{ij} | y_{ij}^c) = \theta' \delta(y_{ij})
$$

Where
\begin{itemize}
\item $y_{ij}^c$ is the complement of $y_{ij}$, i.e. all dyads in the network other than $y_{ij}$
\item $y_{ij}^+$ as the same network as $y$ except that $y_{ij} = 1$
\item $y_{ij}^-$ as the same network as $y$ except that $y_{ij} = 0$
\item $\delta(y_{ij})$ is given by $g(y_{ij}^+) - g(y_{ij}^-)$ which measures how the sufficient 
statistic $g(y)$ changes if the $(i, j)$th edge is "toggled" on or off.
\end{itemize}

In sum, for each of the following sufficient statistics a $n \times n$ matrix $s$ is constructed. 
The entry $s_{ij}$ indicates how the presence of the edge between $i$ and $j$ changes the network 
statistic, holding the rest of the network constant.
