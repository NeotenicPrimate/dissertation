The debates on the structure and dynamics of science have often been based on abstract ideas rather than 
empirical evidence. However, the recent trend towards digitalization of peer-reviewed scientific articles 
has made it possible to test these theories using empirical data. This provides a valuable opportunity to 
advance our understanding of the structure and dynamics of science.

These theories ask a range of questions about the nature of science and its relationship to other fields 
of knowledge. Some of the key questions they address include: 
\begin{itemize}
    \item Are there regular patterns in the way that the structure of science changes?
    \item Is science a distinct field of knowledge or is it embedded in larger domains, is the 
    knowledge generated by scientific methods fundamentally different from other types of knowledge?
    \item Are there clear distinctions between science and non-science or between different types of 
    science (e.g. "hard" and "soft" sciences)? 
\end{itemize}

This paper aims to make three types of contributions. The first set of contributions is theoretical 
and involves synthesizing and unifying different conceptions of scientific development in graph-theoretic 
terms. The second set of contributions is empirical and involves deriving and testing hypotheses from the 
literature on scientific development. The third set of contributions is practical and focuses on developing 
a standard analysis pipeline applicable to a wide variety of fields. This is necessary due to the increasing 
difficulty of coordinating research efforts and bridging "academic silos" in the face of the exponentially 
increasing output of technical knowledge and the fact that researchers only have local knowledge about their 
immediate neighbors.

When discussing the structure of a body of knowledge, I am referring to two main aspects: the citation 
structure and the semantic structure. The citation structure refers to the way in which ideas and information 
are connected through references and citations, while the semantic structure refers to the meaning and relationships 
between the concepts and ideas within the body of knowledge. Both of these aspects of structure are important for 
understanding the organization and development of a particular field of knowledge.

